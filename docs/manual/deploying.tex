\chapter{Deploying}

It's not possible to tell you how to create a deployment of Mongrel2 for your
environment.  You probably have your own special brand of crazy that you are
forced to work with (or that you created) and you'll need to bend Mongrel2 to
your will.  That's fine, Mongrel2 should be fairly flexible to deployments
within a fixed set of design considerations.

What this chapter will do is show you how \emph{I} currently deploy Mongrel2 in
a few scenarios.  This way you can see how it's being done for my small setup,
and then extrapolate from there for your own.  As we come up with more best
deployment and automation we'll update this chapter to reflect them.

Another important thing this chapter will do is give you a set of recipes for
common thing you'll need to do at times.


\section{Mongrel2 Deployment Requirements}

It may seem obvious but I'll go over the things you probably need in order to
continue on in this section:

\begin{description}
\item [Mongrel2] I know, hard to believe, but you actually need to have Mongrel2 installed.
\item [m2sh] Again, not sure why but some folks think they don't need this.  Unless you've
    got your own written you need \shell{m2sh}.
\item [Python] Obviously if you have \shell{m2sh} then you have Python, but some systems
    (like Debian) don't install all of Python.  Make sure your Python setup is good.
\item [root] You'll need root access on your box.  Either through sudo or some other means.
\item [Basic Python Coding] Right now you should be able to do some basic Python.  If you can't 
    code Python then you can probably muddle through this and you may learn something, but learning
    Python will be important later.  If you don't know Python and want a good introduction then I
    will pimp my book \href{http://learnpythonthehardway.org}{Learn Python The Hard Way} which is a
    totally free book that teaches Python as if you know nothing.  Give it a try, or any of the other
    free books I mention, but \emph{don't} read "Dive Into Python" as it is a horrible introduction.
\end{description}

That will get you going at first, and as we go we'll do various other setups to 
get our application working.


\section{The mongrel2.org Configuration}

The configuration we'll go over is the one we use at \url{http://mongrel2.org:6767/}
to run all the demos.  This server is a test server that can go down if we aren't 
careful so the main site doesn't totally run on it yet.  What we'll be configuring
and deploying is:

\begin{description}
\item [mongrel2 on port 6767] A mongrel2 server on port 6767.
\item [deployment chroot] Where this mongrel2 will live.
\item [chat demo] Starting the Python backend and the configuring the directory with the
    frontend code in it.
\item [handlertest] This will be a basic test that just echos back HTTP requests.
\item [tests directory] A simple Dir target that serves files out of tests.
\item [mp3streamer] An mp3streamer applicatio
\end{description}



