\chapter{Hacking}

This chapter is all about making cool things with Mongrel2.  It covers all the non-deployment
features that you get from the browser's side and the handler/backend side of your application.
I'll show you how the chat demo works for the async web sockets.  I'll get into writing your
own handlers using a few other demos.  I'll cover some of the interesting things you can
do with Mongrel2 you can't do with other servers.  Finally, I'll get into practical things,
when to do proxying and when to use a 0MQ handler.

For the majority of this chapter, I'll be using Python, but the demos should translate to
the other languages that are implemented.  I'll periodically show how another language
does one of the demos, so you can get the idea that Mongrel2 is \emph{language agnostic}.
In no way should you take me using Python in this chapter to mean you can't use something
else for your handlers.

Currently supported languages are:

\begin{description}
\item [Python] The directory \file{examples/python} contain the Mongrel2 Python library \ident{m2py}.
\item [Ruby] Probably the most extensively supported language, with good Rack support, by \href{http://github.com/perplexes/m2r}{perplexes on github}.
\item [C++] C++ support by \href{http://github.com/akrennmair/mongrel2-cpp}{akrennmair on github}.
\item [PHP] PHP support by \href{http://github.com/winks/m2php}{winks on github}.
\item [C] You can also write handlers in C using the Mongrel2 library, but it's really rough, and not recommended yet.  A C library will come, though.
\item [Others?] \href{http://zeromq.org}{ZeroMQ} supports Ada, Basic, C, C++, Common Lisp, Erlang, Go, Haskell, Java, Lua, .NET, Objective-C, ooc, Perl, PHP, Python, and Ruby, so after reading this chapter you can easily write handlers in any of those languages too.
\end{description}

However, no matter how many languages Mongrel2 supports, you will still have applications that
can't fit into 0MQ handlers and just work better as classic web apps, either because you've
already written them and have existing infrastructure, or because of some architectural issues
that require it to run traditionally.  Because of that, Mongrel2 supports \emph{HTTP proxying},
which allows you to route requests to basic web server backends that don't support 0MQ.

\begin{aside}{What About FastCGI/AJP/CGI/SCGI/WSGI/Rack?}
Nothing prevents you from writing your own connector between Mongrel2 and your
deployment protocol of choice.  If you need to run FastCGI or AJP in your environment,
then your best bet is to just make a handler that translates Mongrel2 requests
to the protocol you need and back.  The Mongrel2 format is very easy to parse and
translate, so you should be able to do it with no problem.  The Ruby library already supports
Rack as an example, and Python will support WSGI soon.

However, Mongrel2 itself doesn't support any of these directly.  Doing so would bring
back the language specific infections that cause other web servers to go south.  The
design of most of these protocols tends to be either before the modern web, or specific
to one particular language.  Instead of trying to cater to all the possible languages
out there, Mongrel2 just gives the tools to connect to it yourself.
\end{aside}


\section{Front-end Goodies}

Mongrel2 supports your standard web server features like serving files, routing requests to
another HTTP server, multiple host matching, good 304 support, and just generally being able
to interact with a browser like normal.  You've seen most of these features as you setup and
deployed a Mongrel2 configuration, but let's go through some of them in more detail so you
know what's possible.


\subsection{HTTP}

Mongrel2 uses the original Mongrel parser that powers quite a few other web servers and large,
successful websites.  This parser is rock solid, dead accurate, and by design blocks a lot of
security attacks.  For the most part you don't have to worry about this and just need to know
Mongrel2 is using the same stable HTTP processing that has been working great for many years.

Another way to put this is if Mongrel2 says your request is invalid, it most definitely is.

\begin{aside}{Idiots and RFC Implementers}
I don't know why, but people who implement RFCs pick up very weird cargo cult
beliefs peddled by the people who write the standards.  In HTTP it was two things
which the creators of HTTP have actually back-peddled on:  Accept everything, and
keep-alives with pipe-lines.

The truth is, if you want a secure server of \emph{any} kind, blindly accepting
every single thing any idiot sends you is going to open your server up to a
huge number of attacks.  If you look at every attack on existing HTTP servers
you'll find that about 80\% of them are exploiting ambiguous parts of the HTTP
grammar to pass through malicious content or overflow buffers.  In Mongrel2 we
use a parser that rejects invalid requests from first basic principles using
technology that's 30 years old and backed by solid mathematics.  Not only does
Mongrel2 reject bad requests, but it can tell you \emph{why} the request was
bad, just like a compiler.  This doesn't mean Mongrel2 is ruthless, but it
definitely doesn't tolerate ambiguity or stupidity.

Mongrel2 completey supports keep-alives because now, since it's not using Ruby
\emph{at all} it can scale up beyond 1024 file descriptors.  Ruby was limited
in the number of open files a process could have, so the original Mongrel had
to break keep-alive and kill connections in order to save itself from greedy
browsers that never close them.  Mongrel2 doesn't have this limitation, so it
uses full keep-alives and has a dead accurate state machine to manage them correctly.

Where problems come in is with pipe-lined requests, meaning a browser sends a bunch
of requests in a big blast, then hangs out for all the responses.  This was such a
horrible stupid idea that pretty much everone gets it wrong and doesn't support it
fully, if at all.  The reason is it's much too easy to blast a server with a ton
of request, wait a bit so they hit proxied backends, and then close the socket.  The
web server and the backends are now screwed having to handle these requests which will
go nowhere.

Mongrel2 does \emph{not} support pipe-lined requests.  It sends one, and waits for the
reponse, and if you want more, then tough.  Screw you because it has \emph{no} advantage
for Mongrel2 and dubious advantages to you.  It is simply one more attack vector for
the server and is rejected outright.

These two things are rejected outright by Mongrel2 simply because they are stupid ideas
and in 2010 nobody should be writing clients so badly that they need these features.
\end{aside}

\subsection{Proxying}

You've already seen configurations that have the Proxy routes working, so it should
be easy to understand what's going on.  You just create routes to backends that are
HTTP servers and Mongrel2 shuttles requests to them, then proxies responses back.

The Proxying support in Mongrel2 is accurate, but it's not very capable right now.  For
example, there's not round-robin backend selection, or page caching, or other things you
might need for more serious deployments.  Those features will come eventually, though.

What you do get with Mongrel2's proxying, though, is a dead accurate way of slicing up
your application by routes.  Other web servers make you go through great pain in order
to have some URLs go to a proxy and others go to handlers or directories.  They make you
use odd ``file syntax'', weird pseudo-turing logic if-statements, and other odd hacks
to get flexible route selection.  They also tend to not maintain keep-alives properly
between proxy requests and other requests.

Mongrel2 uses the exact same routing syntax for all backends and has no distinction between
them.  It also properly does keep-alives for as long as it is efficient to do so.

\begin{aside}{Proxying And 0MQ Handlers Are Like mod\_*}
A quick note for people coming from other web servers.  If you use nginx then you are probably
familiar with the concept of proxying to a ``backend'' like Ruby on Rails or Django.
If you use PHP or another language, you may be used to a system like \ident{mod\_php} which 
manages your code for you and reloads when you make changes.
If you use Apache, then you probably think in terms of ``virtual hosts'' and ``mod\_rewrite rules''.

In Mongrel2 all the same concepts are there, it's just cleaned up.  If you want Mongrel2
to ``nginx/mod\_rewrite style'' talk to another backend web server, then that's Proxying.
If you want to have fast backend handlers then that's 0MQ Handlers.

We really don't have anything like mod\_php because the whole idea of embedding a programming
language runtime inside Mongrel2 would defeat the point of making it language agnostic.
\end{aside}

\subsection{WebSockets}

Mongrel2 does not support WebSockets because the original protocol was a complete
ugly hack with security holes galore.  They've since fixed the entire protocol
and we'll be implementing the \href{http://tools.ietf.org/html/draft-ietf-hybi-thewebsocketprotocol-07}{hybi-07}
version of the protocol in the 1.7 or 1.8 release.


\subsection{JSSocket}

The Mongrel2 chat demo uses JSSocket to do its magic, and it works great, but it requires
Flash and, oh, man, do I absolutely hate Flash.  However, it works, and works now, and works in every
browser, even really old, busted ones.  That means it's the first thing we implemented and
the one we'll keep for a while until it proves itself not useful.  The chat demo we'll
cover will show you how to hook this up for fast async messaging and presence detection.


\subsection{Long Poll}

Mongrel2 just works as if everything is an HTTP long poll, it's just that normal request/responses
are super fast long polls.  For the most part you don't even need to know this exists; it's just
how things are and they make perfect sense.  You get requests from a certain server with a
certain connected identity, and then you send stuff to that target.  That's it.  If you send it
one response, or a stream of them, or setup a long poll configuration, then that's up to you.


\subsection{Streaming}

Because everything in Mongrel2 is asynchronous, and it allows you to target any connected listeners
from your handlers, even with partial messages, you can easily do efficient streaming applications.  ZeroMQ
is an incredibly efficient transport mechanism, and with it you can send tons of information to many
browsers or clients at once.  This means streaming video and MP3 streams to listeners is very
trivial.  We'll cover the mp3stream example where you get to see a simple implementation of the ICY
MP3 streaming protocol.

\subsection{N:M Responses}

What makes streaming, async messaging, and long poll designs so efficient in Mongrel2 is that you can send
\emph{one} message and target up to 128 clients with that one message.  This means sending large scale replies
to many browsers requires less copying of the message and less transports.

In addition to this, you can setup Mongrel2 with the help of some 0MQ to send
one request from a browser to as many target handlers as you like.  You can
even send them messages using \href{http://code.google.com/p/openpgm/}{OpenPGM}
for sending UDP messages reliably to clusters of computers.

This means that Mongrel2 is the only web server capable of sending one request
from a browser to N backends at once, and then return the replies from these
handlers to M browsers.  Not exactly sure what you could write with that, but
it's probably something really damn cool.

\subsection{Async Uploads}

Mongrel2 also solves the problem of large uploads choking your server
because you can't stop them before they're complete.  Mongrel2 will stream
large requests to temporary files, but it sends your handlers an initial
``upload started'' message.  When the upload is done, you get a final ``upload 
finished'' message.  If, at any time, you want to kill the upload, you just
send a 0-length reply (the official KILL MESSAGE) and the whole thing is
aborted and cleaned up.


\section{Introduction to ZeroMQ}

The ZeroMQ folks have finally written a decent manual for ZeroMQ which you should
probably read.  I recommend you read the \href{http://zguide.zeromq.org/page:all}{``0MQ - The Guide''}
as your introduction to 0MQ.

\section{Handler ZeroMQ Format}

You've read the \href{http://zguide.zeromq.org/page:all}{0MQ Guide} and now you're
ready to see how Mongrel2 talks to your handlers with it.  I won't really call this a ``protocol'',
since ZeroMQ is really doing the protocol, and we just pull fully baked messages out of it.  Instead,
this is just a format, as if you got strings out of a file or something similar.  This message
format is designed to accomplish a few things in the simplest way possible:

\begin{enumerate}
\item Be usable from languages that are statically compiled or scripting languages.
\item Be safe from buffer overflows if done right, or easy to do right.
\item Be easy to understand and require very little code.
\item Be language agnostic and use a data format everyone can accept without complaining
    that it should be done with their favorite\footnote{Except Erlang guys, 'cause they'll always
    complain that everything's not in Erlang}.
\item Be easy to parse and generate inside Mongrel2 \emph{without} have to parse the entire message
    to do routing or analysis.
\item Be useful within ZeroMQ so that you can do subscriptions and routing.
\end{enumerate}

To satisfy these features we use different types of ZeroMQ sockets (soon to be configurable),
a request format that Mongrel2 sends and a response format that the handlers send back.  Most
importantly, there is \emph{nothing about the request and response that must be connected}.  In most
cases they will be connected, but you can receive a request from one browser and send a response
to a totally different one.

\subsection{Socket Types Used}

First, the types of ZeroMQ sockets used are a \ident{ZMQ\_PUSH} socket
for messages from Mongrel2 to Handlers, which means your Handler's receive
socket should be a \ident{ZMQ\_PULL}.  Mongrel2 then uses a
\ident{ZMQ\_SUB} socket for receiving responses, which means your Handlers
should send on a \ident{ZMQ\_PUB} socket.  This setup
allows multiple handlers to connect to a Mongrel2 server, but only
one Handler will get a message in a round-robin style.  The PUB/SUB reply
sockets, though, will let Handlers send back replies to a cluster of
Mongrel2 servers, but only the one with the right subscription will
process the request.\footnote{The types of sockets used will be configurable
in later version}

In the various APIs we've implemented, you don't need to care about this.
They provide an abstraction on top of this, but it does help to know it
so that you understand why the message format is the way it is.

This leads to rule number 1:

\begin{quote}
\emph{Rule 1:} Handlers receive with PULL and send with PUB sockets.
\end{quote}

\subsection{UUID Addressing}

Do you remember all those UUIDs all over the place in the configuration files?
They may have seemed odd, but they identify specific server deployments and
processes in a cluster.  This will let you identify exactly which member of a
cluster sent a message, so that you can return the right reply.  This is the
first part of our protocol format and it results in the next rule 2:

\begin{quote}
\emph{Rule 2:} Every message to and from Mongrel2 has that Mongrel2 instance's
UUID as the very first thing.
\end{quote}

\subsection{Numbers Identify Listeners}

You then need a way to identify a particular listener (browser, client, etc.)
that your message should target, \emph{and} Mongrel2 needs to tell you who is
sending your handler the request.  This means Mongrel2 sends you just one
identifier, but you can send Mongrel2 a list of them.  This leads to rule 3:

\begin{quote}
\emph{Rule 3:} Mongrel2 sends requests with one number right after the server's
UUID separated by a space.  Handlers return a \emph{netstring} with a list of
numbers separated by spaces.  The numbers indicate the connected browser the
message is to/from.
\end{quote}

In case you don't know what a netstring is, it is a very simple way to encode a
block of data such that any language can read the block and know how big it is.
A netstring is, simply, \verb|SIZE:DATA,|. So, to send ``HI'', you would do
\verb|2:HI,|, and it is \emph{incredibly} easy to parse in every language, even
C.  It is also a fast format and you can read it even if you're a human.


\subsection{Paths Identify Targets}

In order to make it possible to route or analyze a request in your handlers
without having to parse a full request, every request has the path that
was matched in the server as the next piece.  That gives us:

\begin{quote}
\emph{Rule 4:} Requests have the path as a single string followed by a
    space and \emph{no paths may have spaces in them}.
\end{quote}


\subsection{Request Headers And Body}

We only have two more rules to complete the message format.

\begin{quote}
\emph{Rule 5:} Mongrel2 sends requests with a \ident{netstring} that contains a
JSON hash (dict) of the request headers, and then another \ident{netstring}
with the body of the request.
\end{quote}

Then there's a similar rule for responses:

\begin{quote}
\emph{Rule 6:} Handlers return just the body after a space character.  It can be \emph{any}
    data that Mongel2 is supposed to send to the listeners.
\end{quote}

HTTP headers, image data, HTML pages, streaming video\ldots You can also send as
many as you like to complete the request and any handler can send it.


\subsection{Complete Message Examples}

Now, even though we laid out all of this as a series of rules, the actual code to implement
these is very simple.  First here's a simple ``grammar'' for how a request that
gets sent to your handlers is formatted:

\begin{Verbatim}
UUID ID PATH SIZE:HEADERS,SIZE:BODY,
\end{Verbatim}

That's obviously a much simpler way to specify the request than all those
rules, but it also doesn't tell you why.  The above description, while
boring as hell, tells you why each of these pieces exist.

To parse this in Python we simply do this:

\begin{code}{Parsing Mongrel2 Requests In Python}
<< d['docs/manual/inputs/parsing_mongrel2_reqs.py|pyg|l'] >>
\end{code}

This is actually all of the code needed to parse a request, and is
fairly the same in many other languages.  If you look at the file
\file{examples/python/mongrel2/request.py}, you'll see a more complete
example of making a full request object.

A response is then just as simple and involves crafting a similar
setup like this:

\begin{Verbatim}
UUID SIZE:ID ID ID, BODY
\end{Verbatim}

Notice I've got three IDs here, but you can do anywhere from 1 up to 128.  Generating
this is very easy in Python:

\begin{code}{Generating Responses}
<< d['docs/manual/inputs/generating_responses.py|pyg|l'] >>
\end{code}

That, again, is all there is to it.  The \ident{send} method is the
one doing the real work of crafting the response, and the \ident{deliver}
method is just using \ident{send} to do all the the target idents
joined with a space.

\subsection{TNetStrings Alternative Protocol}
During the 1.6 development, it became clear that we needed a sort of ``internal''
protocol for some new Mongrel2 features.  This internal protocol should be
able to store all the same things that JSON can, but also store exact binary
data.  This came about because we want to send raw data to handlers and
other parts of the system like the control port, but JSON involved too
much work to parse and deal with that.  We also did various analyses and
found that much of our time was spent just generating JSON.

What we did, then, is create a small modification to netstrings that ``tags''
each element with its type.  We did this by changing the (fairly useless)
trailing `,' character so that it signified the type of what it contained.
Types can be any of the main data types that JSON has (dicts, lists, integers, etc.),
except that ``strings'' are now entirely raw binary strings, with no
definition about whether they hold anything other than 8-bit octets.

We also made the design so it was backward compatible with netstrings.
This lets us use it to directly parse a zeromq message from anyone, and
it will work whether it's a TNetString-style nested structure, or just
a string with JSON in it.

The end result is a simple specification at \href{http://tnetstrings.org}{http://tnetstrings.org}
which encodes a na\"{\i}ve parser that anyone can copy to other languages easily.
Many other people implemented the protocol and it looks like you can do
it in every language in about 100 lines of code.  Implementing a version
with more performance (since every language needs tricks) seems to take
about 500-1000 lines of code.

Mongrel2 now supports either TNetStrings or JSON as defined above, on the
fly, and without any modification to existing handlers.  Internally, Mongrel2
uses TNetStrings to create its internal control port protocol, which makes
working with Mongrel2 programatically even easier.

To demonstrate this, here's the new code for parsing a request in Python:


\begin{code}{Parsing TNetStrings Requests In Python}
<< d['docs/manual/inputs/parsing_reqs_tnetstrings.py|pyg|l'] >>
\end{code}

Our tests also show that TNetStrings are a good compromise between
speed and ease of parsing.  They're hard to get wrong in parsing, easy
to write out, and faster than many other protocols out there.  The few
that are faster are also much, much, harder to parse and more error
prone.  In our tests, we've found that TNetStrings in Python can be
faster than Python's own pickle format when we use a C extension.

The most important point about TNetStrings, though, is how it opens up
Mongrel2 for even more control and automation.

\subsection{Python Handler API}

Instead of building all of this yourself, I've created a Python library
that wraps all this up and makes it easy to use.  Each of the other
libraries are designed around the same idea and should have a similar
design.  To check out how to use the Python API, we'll take a look at
each of the demos that are available.  These are the same demos you
ran in the previous section to create a sample deployment.

For the Python API, you may want to start by looking at two very small files that should be able to understand quickly:
\file{examples/python/mongrel2/request.py} and
\file{examples/python/mongrel2/handler.py}.


\section{Basic Handler Demo}

The most basic handler you can write is in the \file{examples/http\_0mq/http.py} file
and it just the simplest thing possible:\footnote{This is the same code as the original
file, but with extraneous prints removed for simplicity.}

\begin{code}{http.py example}
 << d['examples/http_0mq/http.py|pyg|l'] >>
\end{code}

All this code does is print back a simple little dump of what it received, and
it's not even a valid HTML document.  Let's walk through everything that's going on:

\begin{enumerate}
\item Import the \ident{handler} module from \ident{mongrel2} and \ident{json}.  The \ident{json} module is
    really only used for logging.
\item Establish the UUID for our handler, and create a connection.  It's not \emph{really} a connection
    but more of a ``virtual circuit'' that you can just pretend is a connection.  It's using all ZeroMQ and
    the protocol we just described to create a simple API to use.
\item Go into a while loop forever and recv request objects off the connection.
\item One type of special message we can get from Mongrel2 is a ``disconnect'' message, which tells you that
    one of the listeners you tried to talk to was closed.  You should either ignore those and read
    another, or update any internal state you may have.  They can come asynchronously, and for the most
    part you can ignore them unless you need to keep them open as in, say, a chat application or streaming.
\item Craft the reply you're going to send back, which is just a dump of what you received.
\item Send this reply back to Mongrel2.  Notice the subtle difference where you include the \emph{req} object
    as part of how you reply?  This is the major difference between this API and more traditional
    request/response APIs in that you need the request you are responding to so that it knows where to send
    things.  In a normal socket-based server this is just assumed to be the socket you're talking about.
\end{enumerate}

This is all you need at first to do simple HTTP handlers.  In reality, the \ident{reply\_http} method is
just syntactic sugar on crafting a decent HTTP response.  Here's the actual method that is crafting these replies:

\begin{code}{HTTP Response Python Code}
<< d['docs/manual/inputs/http_response_python_code.py|pyg|l'] >>
\end{code}

Which is then used by \ident{Connection.reply\_http} and
\ident{Connection.deliver\_http} to send an actual HTTP response.  That
means all this is doing is creating the raw bytes you want to go
to the real browser, and how it's delivered is irrelevant.  For example,
the \ident{deliver\_http} method means that, yes, you can have one
handler send a single response to target \emph{multiple} browsers
at once.


\section{Async File Upload Demo}

Mongrel2 uses an asynchronous method of doing uploads that helps you 
avoid receiving files you either can't accept or shouldn't accept.  It does
this by sending your handler an initial message with just the headers, streaming
the file to disk, and then a final message so you can read the resulting file.
If you don't want the upload, then you can send a kill message (a 0 length message)
and the connection closes, and the file never lands.

The upload mechanism works entirely on content length, and whether the file
is larger than the \ident{limits.content\_length}.  This means if you don't
want to deal with this for most form uploads, then just set \ident{limits.content\_length}
high enough and you won't have to.

However, if you want to handle file uploads or large requests, then you add
the setting \ident{upload.temp\_store} to a \ident{mkstemp} compatible path
like \file{/tmp/mongrel2.upload.XXXXXX} with the XXXXXX chars being replaced
with random characters.  It doesn't have to /tmp either, and can be any store
you want, network disk, anything.

Here's an example handler in \file{examples/http\_0mq/upload.py} that shows
you how to do it:

\begin{code}{Async Upload Example}
<< d['examples/http_0mq/upload.py|pyg|l'] >>
\end{code}

You can test this with something like
\verb|curl -T tests/config.sqlite http://localhost:6767/handlertest| to upload a big file.

What's happening is the following process:

\begin{enumerate}
\item Mongrel2 receives a request from a browser (or curl in this case) that is greater than \ident{limits.content\_length} in size.  It actually doesn't read all of it yet, only about 2k.
\item Mongrel2 looks up the \ident{upload.temp\_store} setting and makes a temp file there to write the contents.  If you don't have this setting then it aborts and returns an error to the browser.
\item Mongrel2 sees that the request is for a Handler, so it crafts an initial request message.  This request message has all the original headers, plus a \ident{X-Mongrel2-Upload-Start} header with the path of the expected tmpfile you will read later.
\item Your handler receives this message, which has no actual content, but the original content length, all the headers, and this new header to indicate an upload is starting.
\item At this point, your handler can decide to kill the connection by simply responding with a kill message, or even with a valid HTTP error reponse then a kill message.
\item Otherwise your handler does nothing, and Mongrel2 is already streaming the file into the designated tmpfile for this upload.
\item When the upload is finally saved to the file, it \emph{adds} a new header of \ident{X-Mongrel2-Upload-Done} set to the same file as the first header.  Remember that \emph{both} headers are in this final request.
\item Your handler then gets this final request message that has both the \ident{X-Mongrel2-Upload-Start} and \ident{X-Mongrel2-Upload-Done} headers, which you can then use to read the upload contents.  You should also make sure the headers match to prevent someone forging completed uploads.
\end{enumerate}

\begin{aside}{Watch The chroot Too}
Remember, when you run Mongrel2 it will store the file relative to its \ident{chroot} setting.  In testing you probably aren't
running Mongrel2 as root so it works fine.  You just then have to make sure that your handler know to look for the file in the
same place.  So if you have \file{/var/www/mongrel2.org} for your \ident{chroot} and \file{/uploads/file.XXXXXX} then the
actual file will be in \file{/var/www/mongrel2.org/uploads/file.XXXXXX}.  The good thing is you can read the config database
in your handlers and find out all this information as well.
\end{aside}

\section{MP3 Streaming Demo}

The next example is a very simple and, well, kind of poorly implemented
MP3 streaming demo that uses the ICY protocol.  ICY is a really lame
protocol that was obviously designed before HTTP was totally baked
and probably by people who don't really get HTTP.  It works in an odd
way of having meta-data sent at specific sized intervals so the
client can display an update to the meta-data.

The mp3streamer demo creates a streaming system by
having a thread that receives requests for connections, and then
another thread that sends the current data to all currently connected
clients.  Rather than go through all the code, you can take a look
at the main file and see how simple it is once you get the
streaming thread right:

\begin{code}{Base mp3stream Code}
<< d['examples/mp3stream/handler.py|pyg|l'] >>
\end{code}

Walking through this example is fairly easy, assuming you just trust
that the streaming thread stuff works:

\begin{enumerate}
\item Starts off just like the handler test.
\item We figure out what .mp3 files are in the current directory.
\item Establish a data chunk size of 5k for the ICY protocol and
    make a ConnectState and Streamer from that.  These are the
    streaming thread things found in \file{mp3stream.py} in the same
    directory.
\item We then loop forever, accepting requests.
\item Unlike the handler, we want to remove disconnected clients,
    so we take them out of the STATE when we are notified.
\item If we have too many connected clients, we reply with a failure.
\item Otherwise, we add them to the STATE and then send the initial
    ICY protocol header to get things going.
\end{enumerate}


That is the base of it, and if you point mplayer at it (which is
the only player that works, really) you should hear it play:

\begin{Verbatim}
mplayer http://localhost:6767/mp3stream
\end{Verbatim}

That is, assuming you put some mp3 files into the directory and
started the handler again.

For more on how the actual state and the protocol works, go look
at mp3stream.py.  Explaining it is far outside the scope of this manual,
but the key points to realize are that this is one thread that's
targetting randomly connected clients with a single message to the
Mongrel2 server and streaming it.


\section{Chat Demo}

The chat demo is the most involved demonstration, and I'm kind of getting
tired of leading you by the hand, so you go read the code.  Here's where
to look:

\begin{description}
\item [JavaScript] Look at \file{examples/chat/static/*.js} for the goodies.
    The key is to see how \file{chat.js} works with the JSSocket stuff,
    and then look at how I did \file{app.js} using \file{fsm.js}.
\item [Python] Look at the \file{examples/chat/chat.py} file to see how
    the chat states are maintained and how messages are sent around.
\item [config] The configuration you created in the last chapter
    actually works with the demo, and if you've been following along
    you should have tested it.
\end{description}

Hopefully, you can figure it out from the code, but if not, let me know.


\section{Other Language APIs}

There's at least 10 langauges available for Mongrel2, so check out the
main \href{http://mongrel2.org}{mongrel2.org site} for the full list.

If you want to implement another language, it should be fairly trivial.
Just base your design on the Python API so that it is consistent, but, please,
don't be a slave to the Python design if it doesn't fit the chosen language;
creating a direct translation of the Python is fine at first, but try
to make it idiomatic after that so people who use that language feel at
home and it's easy for them.


\section{Writing Your Own m2sh}

The very last thing I will cover in the section on hacking Mongrel2 is how to
write your own \shell{m2sh} script in your favorite language.  Obviously, if
you're doing this you should probably have a good reason\footnote{Like if
you're a Ruby weenie and C is banned at your company because they like
dogma more than money.}.  What writing your own, or understanding what
\shell{m2sh} is doing will do for you, though, is help you when you start to
think about automating Mongrel2 for your deployments.

Hopefully, I may have motivated you to automate, automate, automate.
This is why we write software.  If I wanted to do stuff manually I'd
go play guitars or juggle.  I write software because I want a computer
to do things for me, and nothing needs this more than managing your systems.

This is why Mongrel2 is designed the way it is, using the MVC model.  It
lets \emph{you} create your own View like m2sh, web interfaces, automation
scripts, and anything else you need to make it easier to manage more.

If you want to write your own \shell{m2sh} then first go have a look at the
Python code in \file{examples/python/config} and the \shell{m2shpy} script that
installs.  This is where each command lives, where the argument parsing is and,
most importantly, the ORM model that works the raw SQLite database.

The next thing to do is to make your tool craft databases and compare the
results to what m2sh does for a similar configuration.  I recommend you make
a database that's ``correct'' with m2sh, and then dump it via \shell{sqlite3}.
After that, use your tool to make your own database, dump it, and then use
\shell{diff} to compare your results to mine.

You can also look at how the C version of m2sh that is installed by default
is written.  It lives in \file{tools/m2sh} and has a completely different
design but does nearly the same things.  If you know C then this comparing
the two is also educational.

Finally, you'll need to look at two base schema files:
\file{src/config/config.sql} and \file{src/config/mimetypes.sql}, where
the database schema is created and the large list of mimetypes that
Mongrel2 knows is stored.\footnote{Incidentally, if you want to add one,
that's the table to put it in.}  Your tool should be able to use this
SQL to make its database, or at least know what it does.

If you do something cool with all of this, let us know.


\section{Config From Anything: Experimental}

As of v1.7 Mongrel2 has the ability to configure itself directly from a
loadable module that you can define.  The feature is very new and probably
not safe to use quite yet, but I'm documenting it here so that people
can start playing with it and then giving me feedback on how to use it.

The first thing to look at is the null.so module in \file{tools/config\_modules/null.c}
which lays out a bare config module that automatically fails.  This module was
using in unit testing to make sure that Mongrel2 handles some simple invalid
inputs to the configuration system.  Here's the code to the module:

\begin{code}{The null Config Module}
<< d['tools/config_modules/null.c|pyg|l'] >>
\end{code}

You can then get Mongrel2 to load this module directly by passing it as a
fourth parameter to the \file{mongrel2} executable:

\begin{code}{Loading The null Config}
<< d['docs/manual/inputs/null_config_run.sh|pyg|l'] >>
\end{code}

In this run, Mongrel2 detected that you gave it a fourth option and 
loaded that as the module to use for configuring itself.  Normally
it just assumes a sqlite3 database, but now it's going to defer
everything to the null.c code above.  It also passes the 2nd parameter
(the path) and 3rd (the UUID) to the module for the operations it
needs to do.  Mongrel2 also doesn't enforce anything for these strings
other than they were arguments, so you don't have to use any real paths
or UUIDs so long as your module can return the right data.

What you then have to do to make your own config module is:

\begin{enumerate}
\item Copy the null.c file to a new file in \file{tools/config\_modules}.
\item Add your .so to the list of ones to build in \file{tools/config\_modules/Makefile}.
\item Run make to confirm that it builds, then \verb|sudo make install| to make sure it shows up in \file{\$PREFIX/lib/mongrel2/config\_modules}.
\item Start making each function return the right \verb|tns_value_t *| results that
it needs.  Look at src/config/module.c for what is currently being used.
\item Look at \file{tests/config\_tests.c:test\_Config\_load\_module} and write a similar unit test to make sure it works right.
\end{enumerate}

Finally, the protocol that's being used is basically a translation of the sqlite3 tables
defined in the \file{src/config/config.sql} schema into a TNetString data type that
Mongrel2 can understand.  The queries are checked for every error I could think up, and
you should get meaningful error messages about column types.  When it doubt, just
look at \file{src/config/module.c} to see how it's being done and then replicate it exactly.

\begin{aside}{m2sh configuration run}{You're On Your Own}
There's also a way to run the same command using \file{m2sh}, but it's
mostly a convenience to get you started.  If you're doing your own
configuration system it's assumed that you probably aren't using
m2sh and have written your own.  In order to make m2sh work with your
config, we'd have to alter m2sh quite a lot and turn it into a generic
"query the config" tool.  That might happen, but it's not there yet.

Rather than confuse the issue, I'll skip documenting it until a later
release when it's more robust.
\end{aside}

