\chapter{Introduction}

Mongrel2 is a web server.  HTTP requests come in, HTTP responses go out.
Request, response.  There is nothing revolutionary or extravagant in what
Mongrel2 does with a browser, apart from supporting fancy asynchronous socket
protocols.  To the browser, Mongrel2 is just this nice web server that has
WebSockets and Flash Sockets in it.  That's it.

What makes Mongrel2 special is how it satisfies these requests in a \emph{
language agnostic} and \emph{asynchronous} way using a \emph{simple messaging
protocol} to talk to \emph{applications}; not just serve files.  Mongrel2 is also
designed to be incredibly easy to \emph{automatically manage} it as part of your
infrastructure.

Other web servers do some of these things, but they either do them in a bastardized
way or not all of them at once.  Plenty of language specific web servers like Node.js
and Jetty have asynchronous operation, but they're not language agnostic
\footnote{Who the hell wants to code Javascript all day?  Yuck.}.  Other web servers
will let you talk to any language as a backend, but they insist on using HTTP
proxying or FastCGI, which is \emph{not} friendly to asynchronous operations.

Mongrel2 is the only web server I know of that actively tries to focus on these
features as a cohesive whole.


\section{Language Agnostic}

The term "language agnostic" came from people who read about Mongrel2 in the
early days, and it means that Mongrel2 does \emph{not} try to promote any one
language over any others.  Mongrel2 does not care if you run a "Python shop" or
if you're a die hard PHP fan or if you hate PHP and love only Ruby on Rails.
Mongrel2 only knows about HTTP requests, HTTP responses, async messages, and
getting them to your gear to meet those requirements.

Language Agnosticism is the most important feature of Mongrel2 and its entire
purpose stems from the desire to reduce the amount of programming language
religion in the world.  Real people want to get things done, not wanker on
which technology is the best or force other people to use their favorite toys.
Instead, Mongrel2 works to just be great for every language and make it easy
to use what works best for a given problem.


\section{Asynchronous}

Many web servers are "asynchronous" internally, and some force you to
know way too much about how they work internally to get anything done.  What makes
Mongrel2's version of asynchronous messaging different is it extends to
\emph{outside} the Mongrel2 server.  This is a powerful concept that even your backends
can operate asynchronously using simple identification of connected clients.

Other servers assume that every request is received by a browser, then sent to a backend,
and then directly sent out to the client and that's it.  Mongrel2 assumes that there
is a connected client, and it sends requests to backends, but it makes no assumptions
about how those backends respond to the clients.  All it requires is that the backend
application send messages addressed to the client and it will write them on the
socket.

Because of this design Mongrel2 can easily house both classic HTTP clients, keep-alive
style HTTP client, chunked encoding responses, JSSockets, or WebSockets using the same
code.


\section{Message Protocol}

In order to properly do asynchronous messaging in a language agnostic way, Mongrel2
needed a good base protocol that allowed for different messaging styles and worked
with many different languages.  HTTP proxying already does this, although it's not
asynchronous at all.  What gives Mongrel2 its special powers is ZeroMQ, a language-
and transport-mechanism-agnostic messaging system that does \emph{not} require a
centralized messaging server to operate.

Using ZeroMQ lets Mongrel2 talk to a huge number of languages, operate within any
kind of network architecture, and do it with a very simple communication model and
API that most programmers can understand.


\section{Application Oriented}

Web servers today are written as if it was still 1995 and all anyone needs to do
is serve files, maybe some graphics.  Today's web applications are not about serving
files; they're about serving application logic \emph{and} doing it asynchronously.
The advent of the bewildering numbers of ways to hack HTTP into an async messaging
protocol\footnote{Comet, long poll, Juggernaut, etc.} is proof enough that the
pressure is on for web servers to be for applications with highly interactive
interfaces.

Mongrel2 can still serve files just fine.  In fact, it's got very accurate and
easy-to-understand file serving code.  However, Mongrel2 will always be about applications.
Fast, scalable, awesome, asynchronous or synchronous applications that need to use languages
that mere mortals can work with, like PHP.  If there's ever a choice, apps win.

\section{Automated Management}

The language agnostic philosophy even extends to the configuration system where
you can use any language you need to configure it and manage it, as long as the
results are a SQLite3 database Mongrel2 can read and work with to run.  There are
great tools for managing this database already written in Python, but if you hate
Python then you can write anything you want.

This pattern is established with servers like Postfix, Exim, Sendmail, qmail,
and others, that convert configuration files to half-assed SQL databases.
Mongrel2 effectively adopts a Model-View-Controller design for its
configuration system, the same way every web application is designed today.
The Model is a SQLite3 database file, which any programming language can access.
The Controller is a Mongrel2 process that reads this file and sets itself up
accordingly.

The View is a Python script called m2sh that gives you a command line "UI" to
configure and setup the Mongrel2 sqlite model.  It gives you commands for
managing it, crafting configurations, looking at them; the works.

But most importantly, \emph{you can write your own}.  You don't have to wait
for a Mongrel2 developer to craft a configuration file parser for your favorite
language, or use some hack job Nagios Perl junk to automate or scan it.  It's
SQLite3 with a solid, simple schema and well written Python code showing you how
it works.

Nothing stops you from automating the hell out of Mongrel2 with that.


\section{Using This Manual}

This manual is intended to be fun to read, so probably the best way to use it
is to actually \emph{read it}.

I know! Revolutionary, right?  I mean, who has time to read and learn about something
these days.  You just want to get in there and get whatever problem you have done,
now! No time for words.  You just want a straight dump right into your brain so
that you are able to solve all your problems instantly and screw all this talking.

You ever ask yourself if this attitude about not wanting to read and learn is
possibly the reason you always get stuck in emergencies with no time to read
and learn?

Something to think about.

My recommendation is that you go through every page of this manual and do the
stuff in it.  Even if you think you won't need something, because you're not
a programmer, or you're not in operations, you should learn it.  Doing so will
make the parts you do need clearer and give you better ideas for later.


